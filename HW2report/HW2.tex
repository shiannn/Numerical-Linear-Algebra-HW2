\documentclass[12pt]{article}
\usepackage{CJKutf8}
\usepackage{geometry}
\usepackage{listings}
\usepackage{amsmath}
\usepackage{commath}

\newgeometry{vmargin={6mm,10mm}, hmargin={10mm,10mm}}   
\begin{CJK}{UTF8}{bsmi}
\author{b05502087 王竑睿}
\date{}
\title{數值線性代數 HW2}

\begin{document}
\maketitle
\section{}
    \subsection*{利用以下code製造出A矩陣}
        \begin{lstlisting}[language={Matlab}]
            function A = GenA(n)
                A = zeros(n,n);
                for i = 1:n 
                    for j = 1:n
                        if(i==j) 
                            A(i,j) = 1;
                        elseif(j==n)
                            A(i,j) = 1;
                        elseif(i>j)
                            A(i,j) = -1;
                        end
                    end 
                end
            end
        \end{lstlisting}
    \subsection*{利用rand製造b向量}
        \begin{lstlisting}[language={Matlab}]
            b=rand(60,1)
        \end{lstlisting}
    \subsection*{(a)}
        \subsubsection*{利用以下code,以partial pivoting計算Ax=b}
    \begin{lstlisting}[language={Matlab}]
    function x = Gaussian(A,b);
        [row,col]=size(A);
        n = row;
        x = zeros(n,1);
        for k=1:n-1  
            %select row 
            maxRow = -1;
            for i = k:n
                if (A(i,k)>= maxRow)
                    maxRow = A(i,k);
                    maxRowIdx = i;
                end
            end
            %[maxRow, maxRowIdx] = max(A(k:n,k));
            %row change
            A( [k, maxRowIdx], : ) = A( [maxRowIdx, k], : );
            b([k,maxRowIdx]) = b([maxRowIdx,k]);

            for i=k+1:n
                xMultiplier = A(i,k)/A(k,k);
                for j=k:col
                    A(i,j) = A(i,j)-xMultiplier*A(k,j);
                end
                b(i) = b(i)-xMultiplier*b(k);
            end
        end
        % backsubstitution:
        x(n) = b(n)/A(n,n);
        for i=n-1:-1:1
            summation = b(i);
            for j=i+1:n
                summation = summation-A(i,j)*x(j);
            end
            x(i) = summation/A(i,i);
        end
    end
    \end{lstlisting}
        \subsubsection*{Observe of useless result}
            計算 $norm(||Ax-b||)$\\
            得到26.9380\\
            誤差相當大
        \subsubsection*{Perturbed A矩陣的結果}
            \begin{itemize}
                \item 利用rand產生$\delta$,用$A_d=A+\delta$計算$A_dx_d=b$的解
                \item 用partial pivoting計算 $x_d$ 和 $x$ 的差距 $||x_d - x||$,可發現較Complete pivoting大
            \end{itemize}
    \subsection*{(b)}
        \subsubsection*{利用以下code,以complete pivoting計算Ax=b}
        \begin{lstlisting}[language={Matlab}]
    function x = CompleteGaussian(A,b)
        [row,col]=size(A);
        n = row;
        x = zeros(n,1);
        Colname = [1:col];
        for k=1:n-1  
            %select row 
            maxVal = -1;
            for i = k:n
                for j = k:n 
                    if (A(i,j)> maxVal)
                        maxRow = A(i,j);
                        maxRowIdx = i;
                        maxColIdx = j;
                    end
                end
            end
            %row change
            [maxRowIdx maxColIdx]
            A( [k, maxRowIdx], : ) = A( [maxRowIdx, k], : );
            b([k,maxRowIdx]) = b([maxRowIdx,k]);
            A( :,[k, maxColIdx] ) = A( : ,[k, maxColIdx]);
            Colname([k, maxColIdx]) = Colname([maxColIdx, k]);
    
            for i=k+1:n
                xMultiplier = A(i,k)/A(k,k);
                for j=k:col
                    A(i,j) = A(i,j)-xMultiplier*A(k,j);
                end
                b(i) = b(i)-xMultiplier*b(k);
            end
            Colname
        end
        % backsubstitution:
        x(n) = b(n)/A(n,n);
        for i=n-1:-1:1
            summation = b(i);
            for j=i+1:n
                summation = summation-A(i,j)*x(j);
            end
            x(i) = summation/A(i,i);
        end
    end
    \end{lstlisting}
    \subsubsection*{Observe of useless result}
        計算 $norm(||Ax-b||)$\\
        得到1.2091\\
        相較於partial pivoting,較為精確
    \subsubsection*{Perturbed A矩陣的結果}
        \begin{itemize}
            \item 利用rand產生$\delta$,用$A_d=A+\delta$計算$A_dx_d=b$的解
            \item 用complete pivoting計算 $x_d$ 和 $x$ 的差距 $||x_d - x||$,可發現較partial pivoting小
        \end{itemize}
\section{}
    \subsection*{(a)}
        \begin{equation*}
            \begin{aligned}
            Let\ U&=L^{-1}A \\
\Rightarrow ||U|| &\leq ||L^{-1}||||A||\\
\Rightarrow \frac{||L||||U||}{||A||} &\leq ||L^{-1}||||L|| = \kappa(L)\\
            \\
      Same\ Let\ L&=AU^{-1} \\
\Rightarrow ||L|| &\leq ||A||||U^{-1}||\\
\Rightarrow \frac{||L||||U||}{||A||} &\leq ||U^{-1}||||U|| = \kappa(U)\\
            \end{aligned}       
        \end{equation*}\\
            Therefore
            $$\gamma_2 \leq min(\kappa(L),\kappa(U))$$  
    \subsection*{(b)}
        \begin{itemize}
            \item 矩陣的1-norm是 maximum column abolute sum,所以 $\norm{\abs{A}}_1 = \norm{A}_1$
            \item 矩陣的$\infty$-norm是 maximum row abolute sum,所以 $\norm{\abs{A}}_\infty = \norm{A}_\infty$
        \end{itemize}
        \begin{equation*}
            \begin{aligned}
                \gamma_1 &= \frac{\norm{\abs{L}\abs{U}}}{\norm{A}}\\
                         &\leq \frac{\norm{\abs{L}}\norm{\abs{U}}}{\norm{A}}\\
                         &= \frac{\norm{L}\norm{U}}{\norm{A}}\\
                         &= \gamma_2\\
            \end{aligned}
        \end{equation*}
    \subsection*{(c)}
        \textbf{In 2-norm case}
        \begin{equation*}    
            \begin{aligned}
                \gamma_2 &= \frac{\norm{L}_2\norm{U}_2}{\norm{A}_2}\\
                         &= \frac{\norm{L}_2\norm{U}_2}{\norm{LU}_2}\\
                         &= \frac{\norm{L}_2\norm{U}_2}{\norm{U}_2} \qquad \text{(because L is unitary and preserve norm)}\\
                         &= \norm{L}_2\\
                         &= 1 \qquad \text{(2-norm of unitary matrix is 1)}\\
            \end{aligned}
        \end{equation*}
        \begin{equation*}    
            \begin{aligned}
                \gamma_1 &= \frac{\norm{\abs{L}\abs{U}}_2}{\norm{A}_2}\\
                         &\leq \frac{\norm{\abs{L}}_2\norm{\abs{U}}_2}{\norm{A}_2}\\
                         &= \sqrt{n} \times \frac{\norm{\abs{L}}_F\norm{\abs{U}}_F}{\norm{A}_F} \qquad \text{(norm equivalence)}\\
                         &= \sqrt{n} \times \frac{\norm{L}_F\norm{U}_F}{\norm{A}_F} \qquad \text{(Frobenius norm is not influenced by absolute value)}\\
                         &= \sqrt{n} \times \frac{\norm{L}_F\norm{U}_F}{\norm{U}_F} \qquad \text{(because L is unitary and preserve norm)}\\
                         &= \sqrt{n} \times \norm{L}_F\\
                         &= \sqrt{n} \times \sqrt{n}\\
                         &= n\\
            \end{aligned}
        \end{equation*}
        \textbf{In Frobenius-norm case}
        \begin{equation*}    
            \begin{aligned}
                \gamma_2 &= \frac{\norm{L}_F\norm{U}_F}{\norm{A}_F}\\
                         &= \frac{\norm{L}_F\norm{U}_F}{\norm{LU}_F}\\
                         &= \frac{\norm{L}_F\norm{U}_F}{\norm{U}_F} \qquad \text{(because L is unitary and preserve norm)}\\
                         &= \norm{L}_F\\
                         &= \sqrt{n} \qquad \text{(Frobenius-norm of unitary matrix is $\sqrt{n}$)}\\
            \end{aligned}
        \end{equation*}
        \begin{equation*}    
            \begin{aligned}
                \gamma_1 &= \frac{\norm{\abs{L}\abs{U}}_F}{\norm{A}_F}\\
                         &\leq \frac{\norm{\abs{L}}_F\norm{\abs{U}}_F}{\norm{A}_F}\\
                         &=\frac{\norm{L}_F\norm{U}_F}{\norm{A}_F} \qquad \text{(Frobenius norm is not influenced by absolute value)}\\
                         &=\gamma_2\\
                         &= \sqrt{n}\\ 
            \end{aligned}
        \end{equation*}
    \subsection*{(d)}
        \subsubsection*{verify the result in (a)}
            \begin{itemize}
                \item kappaL\_norm1 = 120
                \item kappaU\_norm1 = 3
                \item gamma2\_norm1 = 3
                \item kappaL\_norm2 = 77.0072
                \item kappaU\_norm2 = 3.6731
                \item gamma2\_norm2 = 3.0474
                \item kappaL\_normInf = 120
                \item kappaU\_normInf = 6
                \item gamma2\_normInf = 3
            \end{itemize}
            所以,$\gamma_2\leq \min(\kappa(L),\kappa(U))$
        \subsubsection*{verify the result in (b)}
            \begin{itemize}
                \item gamma1\_norm1 = 2.9667
                \item gamma2\_norm1 = 3
                \item gamma1\_normInf = 2.9667
                \item gamma2\_normInf = 3
            \end{itemize}
            所以,$\gamma_1\leq \gamma_2$, for 1-norm, $\infty$-norm
\newpage
\section{}
    \subsection*{(a)}
    \[  A=
        \left[
        \begin{array}{cccc}
            0.5 & 0.5 & 0 & 0\\
            0 & 0.5 & 0.5 & 0\\
            0 & 0 & 0.5 & 0.5
        \end{array}
        \right]
    \]
    \subsection*{(b)}
        令variable數n為20, k為6
        \[  \fontsize{5}{2}
            A=
            \left[
            \begin{array}{cccccccccccccccccccc}
                0.166 & 0.166 & 0.166 & 0.166 & 0.166 & 0.166 & 0 & 0 & 0 & 0 & 0 & 0 & 0 & 0 & 0 & 0 & 0 & 0 & 0 & 0\\
                0 & 0.166 & 0.166 & 0.166 & 0.166 & 0.166 & 0.166 & 0 & 0 & 0 & 0 & 0 & 0 & 0 & 0 & 0 & 0 & 0 & 0 & 0\\
                0 & 0 & 0.166 & 0.166 & 0.166 & 0.166 & 0.166 & 0.166 & 0 & 0 & 0 & 0 & 0 & 0 & 0 & 0 & 0 & 0 & 0 & 0\\
                0 & 0 & 0 & 0.166 & 0.166 & 0.166 & 0.166 & 0.166 & 0.166 & 0 & 0 & 0 & 0 & 0 & 0 & 0 & 0 & 0 & 0 & 0\\
                0 & 0 & 0 & 0 & 0.166 & 0.166 & 0.166 & 0.166 & 0.166 & 0.166 & 0 & 0 & 0 & 0 & 0 & 0 & 0 & 0 & 0 & 0\\
                0 & 0 & 0 & 0 & 0 & 0.166 & 0.166 & 0.166 & 0.166 & 0.166 & 0.166 & 0 & 0 & 0 & 0 & 0 & 0 & 0 & 0 & 0\\
                0 & 0 & 0 & 0 & 0 & 0 & 0.166 & 0.166 & 0.166 & 0.166 & 0.166 & 0.166 & 0 & 0 & 0 & 0 & 0 & 0 & 0 & 0\\
                0 & 0 & 0 & 0 & 0 & 0 & 0 & 0.166 & 0.166 & 0.166 & 0.166 & 0.166 & 0.166 & 0 & 0 & 0 & 0 & 0 & 0 & 0\\
                0 & 0 & 0 & 0 & 0 & 0 & 0 & 0 & 0.166 & 0.166 & 0.166 & 0.166 & 0.166 & 0.166 & 0 & 0 & 0 & 0 & 0 & 0\\
                0 & 0 & 0 & 0 & 0 & 0 & 0 & 0 & 0 & 0.166 & 0.166 & 0.166 & 0.166 & 0.166 & 0.166 & 0 & 0 & 0 & 0 & 0\\
                0 & 0 & 0 & 0 & 0 & 0 & 0 & 0 & 0 & 0 & 0.166 & 0.166 & 0.166 & 0.166 & 0.166 & 0.166 & 0 & 0 & 0 & 0\\
                0 & 0 & 0 & 0 & 0 & 0 & 0 & 0 & 0 & 0 & 0 & 0.166 & 0.166 & 0.166 & 0.166 & 0.166 & 0.166 & 0 & 0 & 0\\
                0 & 0 & 0 & 0 & 0 & 0 & 0 & 0 & 0 & 0 & 0 & 0 & 0.166 & 0.166 & 0.166 & 0.166 & 0.166 & 0.166 & 0 & 0\\
                0 & 0 & 0 & 0 & 0 & 0 & 0 & 0 & 0 & 0 & 0 & 0 & 0 & 0.166 & 0.166 & 0.166 & 0.166 & 0.166 & 0.166 & 0\\
                0 & 0 & 0 & 0 & 0 & 0 & 0 & 0 & 0 & 0 & 0 & 0 & 0 & 0 & 0.166 & 0.166 & 0.166 & 0.166 & 0.166 & 0.166\\
            \end{array}
            \right]
        \]
        random取得x並且建構b = Ax\\
        \[
            x=
            \left[
            \begin{array}{c}
                0.4173\\
                0.0497\\
                0.9027\\
                0.9448\\
                0.4909\\
                0.4893\\
                0.3377\\
                0.9001\\
                0.3692\\
                0.1112\\
                0.7803\\
                0.3897\\
                0.2417\\
                0.4039\\
                0.0965\\
                0.1320\\
                0.9421\\
                0.9561\\
                0.5752\\
                0.0598\\
            \end{array}
            \right]
            b=
            \left[
                \begin{array}{c}
                    0.5491\\
                    0.5358\\
                    0.6776\\
                    0.5887\\
                    0.4497\\
                    0.4980\\
                    0.4814\\
                    0.4654\\
                    0.3827\\
                    0.3372\\
                    0.3407\\
                    0.3676\\
                    0.4620\\
                    0.5176\\
                    0.4603\\
                \end{array}
            \right]
        \]
        \subsubsection*{(i)}
            \[
                A^T=QR
            \]
            \[
                \fontsize{7}{2}
                Q=
                \left[
                    \begin{array}{ccccccccccccccc}
                        -0.41 & 0.62 & -0.07 & -0.07 & -0.08 & -0.09 & -0.11 & -0.35 & 0.05 & -0.05 & -0.06 & -0.06 & -0.07 & -0.07 & -0.07\\
                        -0.41 & -0.12 & 0.61 & -0.07 & -0.08 & -0.09 & -0.11 & 0.07 & -0.35 & -0.05 & -0.06 & -0.06 & -0.07 & -0.07 & -0.07\\
                        -0.41 & -0.12 & -0.13 & 0.60 & -0.08 & -0.09 & -0.11 & 0.07 & 0.07 & 0.34 & -0.06 & -0.06 & -0.07 & -0.07 & -0.07\\
                        -0.41 & -0.12 & -0.13 & -0.15 & 0.58 & -0.09 & -0.11 & 0.07 & 0.07 & -0.08 & 0.34 & -0.06 & -0.07 & -0.07 & -0.07\\
                        -0.41 & -0.12 & -0.13 & -0.15 & -0.17 & 0.57 & -0.11 & 0.07 & 0.07 & -0.08 & -0.08 & 0.33 & -0.07 & -0.07 & -0.07\\
                        -0.41 & -0.12 & -0.13 & -0.15 & -0.17 & -0.19 & 0.55 & 0.07 & 0.07 & -0.08 & -0.08 & -0.09 & 0.33 & 0.33 & 0.33\\
                        0.00 & -0.74 & -0.07 & -0.07 & -0.08 & -0.09 & -0.11 & -0.35 & 0.05 & -0.05 & -0.06 & -0.06 & -0.07 & -0.07 & -0.07\\
                        0.00 & 0.00 & -0.74 & -0.07 & -0.08 & -0.09 & -0.11 & 0.07 & -0.35 & -0.05 & -0.06 & -0.06 & -0.07 & -0.07 & -0.07\\
                        0.00 & 0.00 & 0.00 & -0.75 & -0.08 & -0.09 & -0.11 & 0.07 & 0.07 & 0.34 & -0.06 & -0.06 & -0.07 & -0.07 & -0.07\\
                        0.00 & 0.00 & 0.00 & 0.00 & -0.75 & -0.09 & -0.11 & 0.07 & 0.07 & -0.08 & 0.34 & -0.06 & -0.07 & -0.07 & -0.07\\
                        0.00 & 0.00 & 0.00 & 0.00 & 0.00 & -0.76 & -0.11 & 0.07 & 0.07 & -0.08 & -0.08 & 0.33 & -0.07 & -0.07 & -0.07\\
                        0.00 & 0.00 & 0.00 & 0.00 & 0.00 & 0.00 & -0.76 & 0.07 & 0.07 & -0.08 & -0.08 & -0.09 & 0.33 & 0.33 & 0.33\\
                        0.00 & 0.00 & 0.00 & 0.00 & 0.00 & 0.00 & 0.00 & 0.84 & 0.05 & -0.05 & -0.06 & -0.06 & -0.07 & -0.07 & -0.07\\
                        0.00 & 0.00 & 0.00 & 0.00 & 0.00 & 0.00 & 0.00 & 0.00 & 0.84 & -0.05 & -0.06 & -0.06 & -0.07 & -0.07 & -0.07\\
                        0.00 & 0.00 & 0.00 & 0.00 & 0.00 & 0.00 & 0.00 & 0.00 & 0.00 & -0.84 & -0.06 & -0.06 & -0.07 & -0.07 & -0.07\\
                        0.00 & 0.00 & 0.00 & 0.00 & 0.00 & 0.00 & 0.00 & 0.00 & 0.00 & 0.00 & -0.85 & -0.06 & -0.07 & -0.07 & -0.07\\
                        0.00 & 0.00 & 0.00 & 0.00 & 0.00 & 0.00 & 0.00 & 0.00 & 0.00 & 0.00 & 0.00 & -0.85 & -0.07 & -0.07 & -0.07\\
                        0.00 & 0.00 & 0.00 & 0.00 & 0.00 & 0.00 & 0.00 & 0.00 & 0.00 & 0.00 & 0.00 & 0.00 & -0.85 & -0.85 & -0.85\\
                        0.00 & 0.00 & 0.00 & 0.00 & 0.00 & 0.00 & 0.00 & 0.00 & 0.00 & 0.00 & 0.00 & 0.00 & 0.00 & 0.00 & 0.00\\
                        0.00 & 0.00 & 0.00 & 0.00 & 0.00 & 0.00 & 0.00 & 0.00 & 0.00 & 0.00 & 0.00 & 0.00 & 0.00 & 0.00 & 0.00\\                       
                    \end{array}    
                \right]    
            \]
            \[
                \fontsize{7}{2}
                R=
                \left[
                    \begin{array}{ccccccccccccccc}
                        -0.41 & -0.34 & -0.27 & -0.20 & -0.14 & -0.07 & 0.00 & 0.00 & 0.00 & 0.00 & 0.00 & 0.00 & 0.00 & 0.00 & 0.00\\
                        0.00 & -0.23 & -0.21 & -0.18 & -0.16 & -0.14 & -0.12 & 0.00 & 0.00 & 0.00 & 0.00 & 0.00 & 0.00 & 0.00 & 0.00\\
                        0.00 & 0.00 & -0.22 & -0.20 & -0.18 & -0.16 & -0.13 & -0.12 & 0.00 & 0.00 & 0.00 & 0.00 & 0.00 & 0.00 & 0.00\\
                        0.00 & 0.00 & 0.00 & -0.22 & -0.20 & -0.17 & -0.15 & -0.14 & -0.12 & 0.00 & 0.00 & 0.00 & 0.00 & 0.00 & 0.00\\
                        0.00 & 0.00 & 0.00 & 0.00 & -0.22 & -0.19 & -0.17 & -0.15 & -0.14 & -0.12 & 0.00 & 0.00 & 0.00 & 0.00 & 0.00\\
                        0.00 & 0.00 & 0.00 & 0.00 & 0.00 & -0.22 & -0.19 & -0.17 & -0.16 & -0.14 & -0.13 & 0.00 & 0.00 & 0.00 & 0.00\\
                        0.00 & 0.00 & 0.00 & 0.00 & 0.00 & 0.00 & -0.22 & -0.20 & -0.18 & -0.16 & -0.15 & -0.13 & 0.00 & 0.00 & 0.00\\
                        0.00 & 0.00 & 0.00 & 0.00 & 0.00 & 0.00 & 0.00 & 0.20 & 0.19 & 0.17 & 0.16 & 0.15 & 0.14 & 0.14 & 0.14\\
                        0.00 & 0.00 & 0.00 & 0.00 & 0.00 & 0.00 & 0.00 & 0.00 & 0.20 & 0.19 & 0.17 & 0.16 & 0.15 & 0.15 & 0.15\\
                        0.00 & 0.00 & 0.00 & 0.00 & 0.00 & 0.00 & 0.00 & 0.00 & 0.00 & -0.20 & -0.18 & -0.17 & -0.16 & -0.16 & -0.16\\
                        0.00 & 0.00 & 0.00 & 0.00 & 0.00 & 0.00 & 0.00 & 0.00 & 0.00 & 0.00 & -0.20 & -0.18 & -0.17 & -0.17 & -0.17\\
                        0.00 & 0.00 & 0.00 & 0.00 & 0.00 & 0.00 & 0.00 & 0.00 & 0.00 & 0.00 & 0.00 & -0.20 & -0.18 & -0.18 & -0.18\\
                        0.00 & 0.00 & 0.00 & 0.00 & 0.00 & 0.00 & 0.00 & 0.00 & 0.00 & 0.00 & 0.00 & 0.00 & -0.20 & -0.20 & -0.20\\
                        0.00 & 0.00 & 0.00 & 0.00 & 0.00 & 0.00 & 0.00 & 0.00 & 0.00 & 0.00 & 0.00 & 0.00 & 0.00 & 0.00 & 0.00\\
                        0.00 & 0.00 & 0.00 & 0.00 & 0.00 & 0.00 & 0.00 & 0.00 & 0.00 & 0.00 & 0.00 & 0.00 & 0.00 & 0.00 & 0.00\\                       
                    \end{array}    
                \right]    
            \]
            \[
                x = A^T(AA^T)^{-1}b = QR^{-T}b 
            \]
            \[
                x = 
                \left[
                    \begin{array}{c}
                        0.4173\\
                        0.0497\\
                        0.9027\\
                        0.9448\\
                        0.4909\\
                        0.4893\\
                        0.3377\\
                        0.9001\\
                        0.3692\\
                        0.1112\\
                        0.7803\\
                        0.3897\\
                        0.2417\\
                        0.4039\\
                        0.0965\\
                        0.1320\\
                        0.9421\\
                        0.9561\\
                        0.5752\\
                        0.0598\\
                    \end{array}
                \right]
            \]
        \subsubsection*{(ii)}
            利用matlab左除 $x = A \backslash b$ 得到\\
            \[
                x =
                \left[
                \begin{array}{c}
                          0\\
                    -0.0101\\
                          0\\
                     3.7715\\
                          0\\
                    -0.4669\\
                    -0.0795\\
                     0.8403\\
                    -0.5335\\
                     2.9380\\
                     0.2894\\
                    -0.5664\\
                    -0.1756\\
                     0.3441\\
                    -0.8063\\
                     2.9587\\
                     0.4512\\
                          0\\
                     0.1579\\
                          0\\
                \end{array}
                \right]
            \]
        \subsubsection*{(iii)}
            \begin{itemize}
                \item 利用QR分解以及左除所得到的x都能夠滿足Ax=b的逼近
                \item 但QR分解和左除得到的x的norm不一樣,QR分解的norm比較短
                \item 左除是在所有滿足Ax=b的x中選擇一個,QR分解則是選擇norm最短的那一個
            \end{itemize}
\end{CJK}
\end{document}